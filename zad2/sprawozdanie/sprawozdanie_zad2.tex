\documentclass[a4paper]{article}
\usepackage[a4paper]{geometry}
\usepackage[utf8]{inputenc}
\usepackage{polski}
\usepackage[polish]{babel}
\usepackage[T1]{fontenc}
\usepackage{graphicx}
\usepackage{verbatim}
\usepackage{float}
\usepackage{minted}
\usepackage{microtype}
\usepackage{shortvrb}
\usepackage{amsmath}
\usepackage{svg}


\title{Projektowanie efektywnych algorytmów - sprawozdanie z ćwiczenia 2}
\author{Marcel Guzik}

\begin{document}
\maketitle

Wykonawca: Marcel Guzik 256317

Prowadzący: dr inż. Jarosław Mierzwa

Grupa: Piątek 15:15 grupa późniejsza

Data realizacji zadania: 05.01.2022

\section{Wprowadzenie}

Celem ćwiczenia było zaimplementowanie algorytmu symulowanego wyżarzania dla
asymetrycznego problemu komiwojażera.

\section{Symulowane Wyżarzanie}

Algorytm symulowanego wyżarzania to algorytm heurystyczny polegający na tym że
na początku akceptujemy pewne rozwiązania które mogą być gorsze aby przejść
większą przestrzeń rozwiązań przez co zmniejszamy szansę na utknięcie w minimum
lokalnym.

Metoda symulowanego wyżarzania składa się z kilku części:

\begin{itemize}
      \item Sposób wybierania sąsiada, czyli operacja move

            Algorytm symulowanego wyżarzania działa poprzez stopniowe polepszanie
            aktualnego rozwiązania poprzez wybieranie rozwiązania sąsiedniego, tj.
            rozwiązania które w sposób minimalny różni się od rozwiązania
            poprzedniego. W problemie komiwojażera każde rozwiązanie to permutacja
            miast, a rozwiązania sąsiednie to rozwiązania w których dowolne dwa
            miasta są zamienione miejscami. Poruszanie się po przestrzeni
            rozwiązań w taki sposób, tj. ciągle przesuwając się do rozwiązań
            sąsiednich, stopniowo polepsza rozwiązanie, w nadziei na doprowadzenie
            w ten sposób do rozwiązania optymalnego.

      \item Temperatura oraz funkcja jej schładzania

            Algorytm Symulowanego Wyżarzania różni się od zwykłego algorytmu
            zachłannego - w algorytmie występuje \textbf{temperatura}. Alogrytm
            zaczyna pracę gdy temperatura znajduje się w swojej maksymalnej,
            startowej wartości - oznacza to że w wypadku gdy koszt nowego
            sąsiedniego rozwiązania jest wyższy niż koszt rozwiązania aktualnego,
            istnieje wysoka szansa że zostanie on zaakceptowany jako nowe
            rozwiązanie. Z kolejnymi iteracjami wartość temperatury zmniejsza się
            (co definiuje funkcja schładzania) i prawdopodobieństwo zakceptowania
            gorszych rozwiązań sukcesywnie obniża się.

            W programie stosowana jest następująca funkcja schładzania:

            \[T_{i+1} = \alpha T_i\]

      \item Funkcja determinująca prawdopodobieństwo na przejście do sąsiedniego
            rozwiązania jeżeli to rozwiązanie jest gorsze od aktualnego

            Prawdopodobieństwo przejścia z lepszego rozwiązania $s$ do
            sąsiedniego, gorszego rozwiązania $s_n$ określa funkcja
            prawdopodobieństwa $P(s, s_n, T)$, która wyznacza prawdopodobieństwo z
            różnicy kosztów obu stanów oraz aktualnej temperatury.

            W programie występuje następująca funkcja prawdopodobieństwa:

            \[P(s, s', T) = e^{-(\frac{s' - s}{T})}\]

      \item Warunek stopu

            Warunkiem stopu w programie jest spadek temperatury poniżej temperatury
            minimalnej $t_{min}$.
\end{itemize}

\section{Kod programu}

Program zrealizowano z wykorzystaniem programowania obiektowego. W tym celu
utworzono trzy klasy: \verb|Tsp|, \verb|TspSolver|, oraz \verb|SaTspSolver|.

\subsection{Tsp}

Klasa \verb|Tsp| zawiera informacje reprezentujące instancję programu
komiwojażera, tj. tablicę sąsiedztwa oraz ilość miast.

\subsection{TspSolver}

Klasa \verb|TspSolver| jest klasą abstrakcyjną, definiującą interfejs
pozwalający na uzyskanie rozwiązania dla danej instancji problemu komiwojażera.
Dziedziczona jest ona przez inne klasy implementujące konkretne algorytmy
rozwiązujące problem.

Jest to przykład wykorzystania wzorca projektowego strategii.

\subsection{SaTspSolver}

Klasa \verb|SaTspSolver| dziedziczy po klasie \verb|TspSolver| i implementuje
funkcję \verb|TspSovler::solve()| rozwiązującą problem komiwojażera używając
algorytmu symulowanego wyżarzania (simulated annealing, SA).

\section{Wyniki}

Program kompilowano i testowano na systemie o następującej specyfikacji:

CPU: Ryzen 5 3600 6c/12t 3.6GHz 384KB/3MB/32MB cache

OS: Linux 5.14.21-2-MANJARO glibc: 2.23

skompilowano: g++ -std=c++20 -g -O -Wall -Wextra -Wpedantic src/main.cpp
-o zad1.out

Dla każdej z przykładowych instancji program został uruchomiony 10 razy.
Rozwiązanie najlepsze wybrano do porównania poniżej.

\begin{table}
      \begin{center}
            \begin{tabular}{|c|c|c|c|c|}
                  \hline
                  Nazwa instancji & limit czasowy & znaleziona ścieżka & najkrótsza ścieżka & błąd procentowy \\
                  \hline
                  ftv47.atsp      & 2m            & 1789               & 1776               & 0,7\%           \\
                  \hline
                  ftv170.atsp     & 4m            & 3166               & 2755               & 14,9\%          \\
                  \hline
                  rbg403.atsp     & 6m            & 2789               & 2465               & 13,1\%          \\
                  \hline
            \end{tabular}
      \end{center}
      \caption{Wyniki algorytmu dla trzech przykładowych grafów}
\end{table}

\begin{figure}[H]
      \centering
      \includesvg[width=0.48\textwidth]{cost_47}
      \includesvg[width=0.48\textwidth]{cost_170}
      \includesvg[width=0.48\textwidth]{cost_403}
      \caption{Wykres kosztu aktualnego cyklu dla numeru iteracji działania
            algorytmu dla trzech powyższych grafów, w kolejności od lewej:
            ftv47.atsp, ftv170.atsp, rbg403.atsp}
\end{figure}

Znalezione ścieżki:

\begin{itemize}
      \item ftv47.atsp

            0 37 38 20 40 22 42 43 41 2 28 27 33 9 6 31 5 30 29 4 24 3 8 11 10 1
            26 47 21 44 19 39 45 16 15 14 35 36 46 13 34 23 7 32 12 17 18 25 0

      \item ftv170.atsp

            0 81 80 79 82 78 72 168 50 84 83 109 114 102 162 123 101 100 99 163
            165 96 95 98 97 105 106 107 69 70 167 67 68 62 57 56 64 65 63 66 61
            59 58 54 55 43 53 52 51 60 71 85 86 87 153 88 89 90 154 91 94 92 93
            166 108 110 111 1 77 73 170 49 48 47 46 44 45 42 41 155 34 33 157 36
            35 156 40 39 38 37 158 32 31 30 28 27 26 29 21 17 16 159 15 14 151
            160 161 148 149 150 25 24 23 22 20 19 18 13 12 11 75 74 76 10 9 8 7
            6 142 152 141 140 139 138 136 137 147 143 144 145 146 135 130 128
            129 124 121 122 120 119 117 116 115 113 104 103 118 125 126 127 164
            131 134 133 132 112 169 5 4 3 2 0

      \item rbg403.atsp

            0 373 47 219 13 62 79 371 157 37 332 248 402 309 215 49 299 188 154
            226 230 184 381 324 194 12 280 32 83 267 1 237 343 292 336 325 326
            265 243 132 390 391 112 241 133 310 367 337 138 190 195 175 327 84
            104 124 162 379 363 374 369 180 266 45 89 302 199 142 250 222 350
            351 316 85 225 229 238 255 392 141 204 291 298 51 92 6 306 270 211
            278 59 140 284 300 244 257 271 179 44 60 66 26 172 96 114 67 385 295
            217 338 346 181 269 20 313 9 218 296 305 31 176 231 35 308 165 33 98
            100 178 359 27 95 314 207 339 110 360 91 289 321 366 34 43 129 201
            240 245 368 164 97 136 333 189 203 150 355 131 193 192 214 349 256
            82 318 65 224 72 90 109 128 48 130 388 111 101 151 312 212 146 361
            293 396 398 378 331 242 297 15 7 152 36 137 389 221 118 232 70 166
            210 148 161 342 354 8 285 387 187 191 375 116 120 108 73 261 262 376
            106 356 209 362 205 134 123 144 382 78 75 153 279 17 16 253 169 196
            281 294 329 348 185 174 126 182 117 239 25 163 121 235 68 334 170 4
            139 30 206 202 383 384 93 247 288 249 301 127 55 335 56 50 54 167 61
            2 386 107 328 370 358 357 273 125 122 228 234 69 88 283 168 86 24
            377 19 105 42 323 159 352 347 149 39 40 52 143 307 38 395 287 10 311
            46 322 282 145 197 220 401 213 71 63 171 259 276 177 319 74 286 315
            380 233 5 87 340 41 272 274 264 290 330 3 64 365 160 320 227 236 57
            303 364 263 353 14 394 304 102 76 155 29 53 186 147 156 77 208 317
            113 251 260 397 268 183 258 275 135 99 103 341 80 246 158 21 22 81
            198 399 252 400 173 345 393 23 94 115 223 254 344 277 58 18 200 216
            119 11 28 372
\end{itemize}

\section{Podsumowanie}

Symulowane wyżarzanie jest algorytmem dobrym dla większych instancji problemu
komiwojażera. Możliwe że nie zwróci rozwiązania optymalnego, jednak dla dużych
wielkości problemu (30 wzwyż) złożoność obliczeniowa rośnie zbyt wysoko aby
uzyskać optymalne rozwiązanie w rozsądnym czasie, zatem dla dużych instancji
będziemy wspomagać się algorytmami opartymi o metaheurystyki, które mogą zwrócić
rozwiązanie nieoptymalne, ale bardzo bliskie optymalnemu.


\end{document}